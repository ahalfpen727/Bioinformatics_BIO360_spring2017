% Bioinformatics Course Materials by David Weisman
% (Weisman@Computer.Org) is licensed under a Creative Commons
% Attribution-NonCommercial-ShareAlike 4.0 International License.

\documentclass[12pt]{article}
\raggedright
\usepackage{url}
\usepackage{fancyhdr}    %% fancyheader is obsolete
\usepackage{booktabs}
\usepackage{array}
\usepackage{framed}
\usepackage{mathptmx}    % package times obsolete, see l2tabuen.pdf
\usepackage[top=1.0in,bottom=1.0in,left=1.0in,right=1.0in]{geometry}

\setlength{\parskip}{1em}
\setlength{\parindent}{0cm}
\setlength{\fboxsep}{0.7cm}

    \raggedbottom                    %less likely orphans and widows
    \addtolength{\topskip}{0pt plus 10pt} % from FAQ
    \clubpenalty=10000
    \widowpenalty=10000
\usepackage[compact]{titlesec}  %compress title vertical whitespace
% \setcounter{secnumdepth}{0}

\newcommand{\TBD}[1]{{\LARGE \fbox{TBD: #1}}}
\newcommand{\course}{BIOL 360/560}

\begin{document}

\pagestyle{fancy}
\fancyhf{}              % Clear all
\fancyhead[L]{\textsf{\textit{Fall 2014}}}
\fancyhead[C]{\textsf{\textit{\course\ Bioinformatics}}}
\fancyhead[R]{\textsf{\textit{Syllabus}}}
\fancyfoot[C]{\textsf{\textit{\thepage}}}

{\centering{\huge{BIOL-360 / BIOL-560 Bioinformatics \par \bigskip Fall 2014}}\par}


\section{Course Information}
\label{sec:course-objectives}



{\setlength{\extrarowheight}{10pt}  % increase table row height
\begin{tabular}{>{\bfseries}l>{\raggedright}p{10cm}}

  Course &  \course\ Bioinformatics, Fall 2014 \tabularnewline
  Credit hours & 3 \tabularnewline
  Class meetings &  Mon Wed Fri, 10:00--10:50, Wheatley W01-0006  \tabularnewline
  Prerequisites & Genetics, Cell Biology, Population Biology, Precalculus   \tabularnewline
  Professor &  Todd Riley\tabularnewline
  Office hours &  Tue and Wed 11:00am - Noon and by appointment  \tabularnewline
  Office & Wheatley W03-019  \tabularnewline
  Phone & 617.287.3236 \tabularnewline
  Mailbox &  In the biology office, W03-021 \tabularnewline
  Blackboard Learn login &  \url{https://umb.umassonline.net} \tabularnewline
  Blackboard help & Info \url{https://login.umassonline.net/boston.cfm}.  \tabularnewline
  Teaching Assistant & Cory Colaneri \url{<ccolaneri1@gmail.com>}  \tabularnewline
  TA Office Hours &  Tue and Thur 3:30 - 4:30pm and by appointment  \tabularnewline
  TA Office &  Wheatley W04-048

\end{tabular}
}

\medskip
\fbox{\parbox{0.9\textwidth}{
    \raggedright
    \smallskip
    \textbf{Blackboard and email are always the best and fastest way to reach me}.

    \smallskip There are two cases:
    \begin{itemize}
    \item For {\textbf{general questions}} about a lecture,   homework,
      schedule, or final project, then it's \mbox{\textit{99\% certain that
          other students have the exact same question}}.   Please \textbf{post
        your general questions} to our \textbf{anonymous} Blackboard class-wide
      discussion list so that the entire class will benefit from   this exchange.
      To post your questions, go to Blackboard and select \mbox{\textbf{Course
          Questions and Answers}}.


    \item For {\textbf{private discussions only}}, please
      contact me via regular email at \url{Todd.Riley@umb.edu}.
    \end{itemize}
  }\smallskip
}

\section{Semester schedule}
\label{sec:schedules}
\begin{minipage}{1.0\linewidth} % keep together

  {Important course dates:  }
  \smallskip

  \begin{tabular}{ll}
    \toprule
    \textbf{Date} & \textbf{Event}\\
    \midrule
    September 3      (Wednesday)             & \course\ First Class                      \\
    September 9      (Tuesday)               & Add/Drop Ends \\
    October 13       (Monday)                & Columbus Day (Holiday) \\
    November 11      (Tuesday)               & Veterans Day (Holiday) \\
    November 19      (Wednesday)             & Pass/Fail \& Course Withdrawal Deadline \\
    November 27 to 30 (Thursday to Sunday)   & Thanksgiving Recess \\
    December 1        (Monday)               & Classes Resume \\
    December 12       (Friday)               & Last \course\ class \\
    December 15 to 19 (Monday to Friday)     & Final Exam Period (no final exam) \\
    December 19       (before end of Friday) &  \course\ Final Project due\\
    \bottomrule
  \end{tabular}
\end{minipage}

\section{Overview and objectives}
\label{sec:objectives}

Bioinformatics is a large and rapidly growing field at the intersection of
biology, computer science, mathematics, and statistics.  \course\ is a
bioinformatics course specifically designed for undergraduate biology majors,
although computer science, math, physics, ECOS, and other majors are certainly
welcome too.


In its very short history, bioinformatics has rapidly grown as
biology and biomedical research have become more quantitative with the need to
analyze exponentially increasing amounts of data in order to answer important
biological questions. Bioinformatics has become an essential investigative
tool. It's as important as the microscope.  The skills you learn in this
course will be highly valuable in graduate school and beyond.  These skills will
greatly benefit students who wish to pursue careers in biotechnology, clinical
medicine, pure research, or in any other biology-associated field.
Bioinformatics itself is a strong career path and advanced bioinformatics skills
are in extremely high demand.


There are three fundamental areas that are driving the bioinformatics fields
today and all three will
be central to
this course.  First, \textbf{high-throughput DNA and RNA sequencing} have become very fast and
inexpensive, and the resulting data gives researchers an immensely valuable resource.
Due to the immense amount of generated sequencing data, scientists have to manage it, cross-link it, and
search it with large computers and databases.  Second, \textbf{molecular
  evolution} occurs through changes in germ-line DNA. Studying
these sequence changes through time provides important clues about evolution itself, as well
as insights into key biological mechanisms such as enzyme catalysis, protein
folding, transcriptional regulation, and cellular signaling.  Third, the field
of \textbf{systems biology} is rapidly growing. Its focus is the system-wide
study of organisms as complex networks of chemical interactions. Successful
analysis of these large and complex networks requires massive
amounts of data and computations, and cannot be analyzed by hand.


Using these three fundamental area as our framework, we will survey several
central areas of bioinformatics, both \textbf{conceptually} and
\textbf{practically}.  The conceptual material relates biological ideas to
bioinformatics algorithms. For example, you will learn about DNA sequence
evolution between individuals and between species, and learn how computer
algorithms can reconstruct an evolutionary tree from those DNA sequences.  At
the practical level, you will learn about bioinformatics tools that are available on the
web, and you'll use these tools in your homework and Final Project.


After completing this course, you will have a basic understanding of key
bioinformatics concepts and practices, and you will be able to apply this
knowledge to new and diverse problems.  As equally important, you will have a
solid foundation for learning new concepts and practices as the bioinformatics field
continues to evolve and expand.


The prerequisites for this course are Genetics, Cell Biology, Population
Biology, and Precalculus.  If you haven't had these prerequisites, or if you're
very rusty in this material, please contact me.  \course\ course does not
require a background in computer science and we won't be doing any programming
\footnote{The lab course associated with \course\ does some introductory
  programming.} \footnote{If you have a computer science background and wish to
  use it on your \course\ final project, please contact me.}.  The mathematics
level will be similar to Genetics Biol-252/254, and we will cover some basic
concepts of probability.


\course\ places a \textbf{strong emphasis on collaborating with your peers}
during class meetings and through the UMass Blackboard system (Section
\ref{sec:online-homew-disc}).  You will work on group homework, and collaborate
on your final projects over Blackboard.  You'll find this structured
collaboration to be very educational, rewarding, and great practice for your
future careers.


\section{Materials}
\label{sec:materials}

\subsection{Textbooks}
\label{sec:textbooks}

The \textbf{required textbook} is \textit{Bioinformatics and
  Functional Genomics}, {2nd edition} by Jonathan Pevsner.  Make
sure to get the 2nd edition (2009); the 1st is much too obsolete.  The book has
an associated web site \url{http://www.bioinfbook.org/} with good slides and
audio lectures by the author.


We'll also read several research and review papers, as well as documentation of
different bioinformatics tools throughout the course.


For biological background and review information, textbooks in genetics, cell
biology, biochemistry, and molecular biology are useful - particular
the chapters on genomics, molecular evolution, mutations, gene
regulation, and protein structure.  Definitely check out the
\textbf{terrific free online biology textbooks} at NCBI
(\url{http://www.ncbi.nlm.nih.gov/books}).  The NCBI
Bookshelf is an amazing resource for biology students.

\subsection{Course web site and computers}
\label{sec:course-web-site}

\textbf{You will need web access throughout the entire semester} for
accessing Blackboard and for accessing bioinformatics tools on the web.  You can
use your own computers or machines in UMB labs.  Macs, PCs, and Linux machines
are all equally sufficient to fulfill class duties.


Note that \course\ is using the new Blackboard Learn 9, which may be different from
the Blackboard you have used in other courses in the past.  Be sure to login at
\url{https://umb.umassonline.net/}.


You will need a supported Web browser to access Blackboard, NCBI, and other
web-based bioinformatics tools. Please check your browser compatibility at:

% \item \url{https://login.umassonline.net/boston.cfm}
 \url{http://www.ncbi.nlm.nih.gov/guide/browsers}

We'll start using Blackboard and iClickers on \textbf{Wednesday, Sept 10} right after the
Add-Drop date, so make sure you have your iClickers ready.


\subsection{Protecting your data}
\label{sec:protecting-your-data}

For your important files, \textit{and especially for your Final Project}, you
should  \textbf{regularly make backup copies!!}  Computers get viruses, laptops get
stolen, hard drives crash, and thumb drives get lost.  Losing important files is
never justification for late or missing work.  To perform backups people often backup files on a USB
flash drive, email files to themselves, or use an online file storage and backup
service.  Whichever approach you take, make sure you have multiple copies of
your work in multiple physical locations. \textbf{Again, perform backups often!}



\section{Grading}
\label{sec:grading}
The \course\ grading system is designed to keep you engaged and involved
throughout the entire semester, instead of cramming last-minute for a couple huge
exams.  If you are genuinely engaged and involved in the homework, discussions,
reading, and classwork, you will receive top grades.  And  \textbf{you'll learn
  more}.


The following sections describe the components that make up the final grade.

\subsection{iClicker questions \& Class participation: 1/3  of final grade}
\label{sec:iclicker-25}

To encourage attendance and lots of in-class participation, we'll heavily use
the iClicker.   You'll get iClicker points for:

\begin{itemize}
\item Participating in classroom group exercises and discussions
\item Answering short questions about the reading
\item Answering short questions about your homework
\item Working out some more detailed quiz questions
\item Answering survey questions to check whether critical concepts need
  further discussion

\end{itemize}

{\textbf{Bring your iClicker to every class and always bring spare batteries!}}
Answers on paper will not be accepted. Read the instructions on the back of the
iClicker so you'll know when the batteries need replacement.


The lowest iClicker grade will be dropped when computing your final iClicker score.

\subsection{Online homework and discussion questions: 1/3 of final grade}
\label{sec:online-homew-disc}

You will work in small groups using Blackboard to discuss concepts,
work through homework questions, and peer-review Final Projects.  You will
need to think carefully, post answers to the questions, and post follow-up messages
to your peers' responses.  Importantly. the questions are open-ended, just like in
the real-world of bioinformatics.


Here is our system:
  \begin{itemize}
  \item Homework runs in 8-day cycles.  Assignments will be posted
    by end of Friday, and the discussion will be closed by 9 PM
    the next Sunday.

  \item Your writing serves multiple purposes: It helps other students
    understand the material; it helps you reinforce your own learning;
    and it brings new ideas into the discussion.  Therefore,
    \textbf{it's very important to write clearly}.  Writing informally
    is perfectly fine.  However, you won't receive any points for writing IM
    style like \textit{omg, imo tps cya}, or for answers like
    \textit{I agree with her}.

  \item Your discussion will benefit by including \textbf{outside materials}
    such as textbooks, bioinformatics web sites, and scientific
    journal papers.  When you mention outside materials, be sure to
    cite them so others can follow your ideas.  Informal citation
    format is fine here.  It's also great to integrate ideas from your
    other courses and labs.


  \item When posting comments about previous messages, always be
    \textbf{constructive and professional}.

  \item You need to \textbf{post your messages within your Blackboard
      group}.  Regular email messages won't receive any points.

  \item You may discuss the questions with anyone outside your
    Blackboard group, but you cannot take their ideas and simply
    reword them.  If you get outside ideas from others, you must
    \textbf{acknowledge their contribution} in your written answer,
    and you must add your own new ideas to the discussion.
  \end{itemize}

  Here is how your homework discussions are graded:

  \begin{itemize}
  \item \textbf{``A'' Discussion (90-100): Distinguished/Outstanding}

    Students earning an ``A'' for discussion activities have
    participated two or more times during the week and have posted
    outstanding information.  ``A'' grade postings:
    \begin{itemize}
    \item  are made in time for others to read and respond
    \item  deliver information that is full of thought, insight, and analysis
    \item make connections to previous or current content or to
      real-life situations
    \item contain rich and fully developed new ideas, connections, or
      applications.
    \end{itemize}

  \item     \textbf{``B'' Discussion (80-89): Proficient}

    Students earning a ``B'' for discussion activities have
    participated at least two times during the week and have posted
    proficient information. ``B'' grade postings:
    \begin{itemize}
    \item  are made in time for others to read and respond
    \item deliver information that shows that thought, insight, and
      analysis have taken place
    \item make connections to previous or current content or to
      real-life situations, but the connections are not really clear
      or are too obvious
    \item contain new ideas, connections, or applications, but they
      may lack depth and/or detail.

    \end{itemize}
  \item     \textbf{``C'' Discussion (70-79): Basic}

    Students earning a ``C'' for discussion activities have
    participated at least one time during the week and have posted
    basic information. ``C'' grade postings:
    \begin{itemize}
    \item  may not all be made in time for others to read and respond
    \item are generally competent, but the actual information they
      deliver seems thin and commonplace
    \item make limited, if any, connections which are often cast
      in the form of vague generalities
    \item contain few, if any, new ideas or applicationsm, but instead are often
      a rehashing or summary of other comments.

    \end{itemize}
  \item \textbf{``D'' and ``F'' Discussion (10-69): Below Expectations}

    Students earning a ``D'' and ``F'' for discussion activities have
    participated at least one time during the week and have posted
    information that was below expectations. ``D'' and ``F'' grade postings:
    \begin{itemize}
    \item  may not all be made in time for others to read and respond
    \item are rudimentary and superficial; there is no evidence of
      insight or analysis
    \item  are impossible to understand, or worse, rude to the other team members
    \item  contribute no new ideas, connections, or applications
    \item  may be completely off topic.

    \end{itemize}


  \end{itemize}
    Here's an example of a homework question:

    \begin{quote}
      Q: Your lab studies how gasoline pollutes the environment, in
      particular, how soil bacteria respond to these toxic
      hydrocarbons.  You ran a traditional 2D-gel proteomic experiment
      measuring \textit{E.~coli} response to gasoline treatments,
      found a protein that differed when treated with gasoline, and
      performed a mass spec on the gel spot to identify the protein.
      Next, you used a software tool to search the mass spec data in
      an \textit{E.~coli} protein database, and it reported several
      possible matches but with extremely low confidence scores.  What
      might be causing these low scores?
    \end{quote}
    Here's an example of a well-written reply post that would get a
    perfect score.  \begin{quote} \setlength{\parskip}{1em}
      \setlength{\parindent}{0cm} One possible problem is
      contamination.  If that occurred, the mass spec data would
      have unexpected peaks, and this could confuse the database
      searching algorithm.  The contamination could have come from
      other types of bacteria or fungi in the sample.  The homework
      question doesn't state whether the experiment was done under
      careful sterile conditions or with regular soil bacteria from
      the ground.

      Another type of contamination came up in the book Bioinformatics
      by Baxevanis, 3'rd Ed.  On p.~467 they mention that
      contamination from trypsin fragmentation could confuse the
      analysis of peptide fragments from the sample.  Foreign protein
      contamination could apply to our homework question, because the
      software isn't expecting mammalian trypsin fragments in a
      bacterial sample!

      In the case of trypsin, the enzyme can cleave itself in a
      process called autolysis.  If this occurs, the mass spec signals from
      those cleaved trypsin fragments could confuse the search algorithm.
      To get around this problem, I found that mass spec experiments
      can use specially-engineered trypsin that's resistant to
      autolysis (see
      \url{http://www.stratagene.com/products/displayProduct.aspx?pid=874}).
      However, it's not clear whether this engineered trypsin completely eliminates all autolysis, or
      just reduces autolysis.  It might reduce autolysis contamination enough so
      the database searching software finds high confidence matches.

      Alternatively, I wonder if the fragment-searching software could
      look for common contaminants like trypsin fragments, remove that
      data from the input, and then search the \textit{E.~coli}
      protein database?  Anyone seen any references to this idea?
    \end{quote}


\subsection{Final Project: 1/3  of final grade}
\label{sec:final-project-30}


Your Final Project will research a gene and its protein using bioinformatics
tools and then write a paper on your findings.  The project will start early in the
semester when you find an interesting gene to research.  Some of the homework
and in-class questions will involve your final project.


Everyone will do their own unique Final Project, but you'll also use your
Blackboard group to get several cycles of feedback as your project progresses.  Your peers
will make suggestions and help bring in new ideas.


Further details on the Final Project will be provided in October.


\subsection{Biol-560 Additional Requirements}
Biol-560 graduate students need to contact me before the end of September to discuss the additional course requirements.

\section{Frequently Asked Questions}
\label{sec:freq-asked-quest}

\begin{itemize}
\item \textbf{On group homework, if somebody in my group is late or does a bad job, will that affect my grade?}

{Never.  No other person's work will hurt your grade.}

\item \textbf{Bio-360 isn't in Blackboard!!  What should I do?}

You probably logged into the old Blackboard.  We're using the new Blackboard
Learn 9, which has a different login page.  Login to the new Blackboard at
\url{https://umb.umassonline.net/}.


\item \textbf{I'm having trouble with Blackboard!!  What should I do?}

  Please call 1-888-300-6920 (24 hours) for help.  You can also try emailing
  \url{bostonsupport@umassonline.net}, but I \textbf{strongly}
  recommend the telephone over email.

  Also try accessing Blackboard from another computer.

\item \textbf{I forgot my iClicker.  Can I email you the answers?}

{No. The best solution to avoid this problem is to ALWAYS keep your iClicker in
  your school bag.  We will use the iClickers heavily, so you need to come up
  with a system that works for you.}


\item \textbf{My iClicker batteries just died!  Can I email you the answers?}

No. Keep a spare set of batteries in your bag.  We'll accommodate if you need a moment to swap batteries.

\item \textbf{A bioinformatics tool seems broken.  Can I have more time for my homework?}

We strongly encourage you to start homework early in the weekly cycle, so that you'll
have plenty of time to get past unexpected problems.  \textbf{If there's a
  problem, definitely post to Blackboard and do it early in the homework
  cycle}.


\end{itemize}


\section{Academic policies}
\label{sec:policies}

The iClicker and in-class participation form a large portion of your final
grade, so you should plan to attend all class meetings.  \textbf{You should
  arrive on time because some iClicker questions happen at the beginning of each
  class.}  You must use only your own iClicker; answering for your absent
friend is not permitted.


\textbf{Please turn off your mobile phone and anything else that beeps or rings before
the class starts.  Texting in class distracts other students, so please don't.}


Academic dishonesty is grounds for automatic failure in this course.  While this
course strongly encourages collaboration, \textbf{all writing must be your own}.
Copying/pasting unattributed text from the web into your assignments is
absolutely plagiarism, and is unacceptable in this course.   \course\ enforces
the UMB plagiarism policy described at
\url{http://www.umb.edu/life_on_campus/policies/code/}.


Blackboard SafeAssign automatically checks text in your homework and Final
Project against the web, journal papers, and other students'
writing. \textbf{Please know that if
  you copy material from these sources, you will probably get caught.}


\section{Resources}
\label{sec:resources}

If you have a disability and feel you will need accommodations in order to
complete course requirements, please contact the Ross Center for Disability
Services (Campus Center 2nd Fl., Room 2010) at 617.287.7430.


\end{document}

%%% Local Variables:
%%% mode: LaTeX
%%% TeX-PDF-mode: t
%%% End:

